%\title{Math 453 HW 9}
\documentclass[addpoints]{exam}

\usepackage{amsmath,amssymb,enumitem,wrapfig,booktabs}
\usepackage{tikz}

\newcommand{\StudentName}{Student Name}
\newcommand{\AssignmentName}{HW 9}

\pagestyle{headandfoot}
\runningheadrule
\firstpageheadrule
\firstpageheader{Math 453}{\StudentName}{\AssignmentName}
\runningheader{Math 453}{\StudentName}{\AssignmentName}
\firstpagefooter{}{}{}
\runningfooter{}{}{}

\printanswers

\begin{document}


Organize your work and show any work that you want credit for. Use full sentences where possible.

\begin{questions}

\question \textbf{CORE M.28}
Use the Subring Test to decide whether each set is a (sub)ring.
\begin{parts}
\part Is $S=\{a+b\sqrt{2}+c\sqrt{3}+d\sqrt{6}\mid a,b,c,d\in\mathbb{Z}\}$ a subring of $\mathbb{R}$?
\part If $U$ is the set of all $2\times 2$ matrices of the form $\begin{pmatrix}x&0\\y&0\end{pmatrix}$, where $x,y\in\mathbb{R}$, is $U$ a subring of $\mathcal{M}_{2\times2}(\mathbb{R})$?
\part If $V$ is the set of all $2\times 2$ matrices of the form $\begin{pmatrix}x&y\\-y&x\end{pmatrix}$, where $x,y\in\mathbb{Z}$, is $V$ a subring of $\mathcal{M}_{2\times2}(\mathbb{Z})$?
\part For a ring $R$, the \emph{center} of $R$ is the set $C$ of all elements $x\in R$ such that $xr=rx$ for all $r\in R$. Is $C$ a subring of $R$?
\end{parts}
\fullwidth{It is recommended that you work through exercises 7, 8, 9, 10, and 11 of Investigation 9 in the book, though they are not assigned as part of the homework.}

\question \textbf{M.29}
\begin{parts}
\part In the ring $\mathbb{Z}_2\oplus\mathbb{Z}_6$ calculate $(1,3)+(1,3)$ and $(1,3)\cdot(1,3)$.
\part In the ring $\mathbb{Z}_3\oplus\mathbb{Z}_4$ calculate $(1,3)+(1,3)$ and $(1,3)\cdot(1,3)$.
\part Find $\text{char}(\mathbb{Z}_2\oplus\mathbb{Z}_6)$ and $\text{char}(\mathbb{Z}_3\oplus\mathbb{Z}_4)$.
\part Find all units in  $\mathbb{Z}_2\oplus\mathbb{Z}_6$ and $\mathbb{Z}_3\oplus\mathbb{Z}_4$.
\part One of the rings $\mathbb{Z}_2\oplus\mathbb{Z}_6$ or $\mathbb{Z}_3\oplus\mathbb{Z}_4$ has the exact same ring structure as $\mathbb{Z}_{12}$. Based off of your answer to parts (c) and (d), which one has the exact same ring structure as $\mathbb{Z}_{12}$?
\end{parts}

\question \textbf{M.30}
Let $S$ and $T$ be subrings of a ring $R$. For each of the questions below, give a proof or a pair of examples (whichever is most appropriate) to justify your answer.
\begin{parts}
\part Is $S\cup T$ always, sometimes, or never a subring of $R$?
\part Is $S\cap T$ always, sometimes, or never a subring of $R$?
\part Is $S\triangle T$ always, sometimes, or never a subring of $R$?
\end{parts}





\end{questions}
\end{document}
