%\title{Math 453 HW 7}
\documentclass[addpoints]{exam}

\usepackage{amsmath,amssymb,enumitem,wrapfig,booktabs}
\usepackage{tikz}

\newcommand{\StudentName}{Student Name}
\newcommand{\AssignmentName}{HW 7}

\pagestyle{headandfoot}
\runningheadrule
\firstpageheadrule
\firstpageheader{Math 453}{\StudentName}{\AssignmentName}
\runningheader{Math 453}{\StudentName}{\AssignmentName}
\firstpagefooter{}{}{}
\runningfooter{}{}{}

\printanswers

\begin{document}


Organize your work and show any work that you want credit for. Use full sentences where possible.

\begin{questions}

\question \textbf{CORE M.18}
Let $n$ be a nonnegative integer, and let $n\mathbb{Z}=\{nx\mid x\in\mathbb{Z}\}$, with addtion and multiplication defined as in $\mathbb{Z}$. 
\begin{parts}
\part Is $n\mathbb{Z}$ a ring? 
\part If so, is $n\mathbb{Z}$ commutative? 
\part Does $n\mathbb{Z}$ have an identity? 
\part Does your answer depend on the value of $n$?
\end{parts}


\question \textbf{CORE M.19}
Let $\mathbb{R}^+$ denote the set of all positive real numbers. For all $x,y\in\mathbb{R}^+$ define 
\[
x\oplus y=xy\text{\quad and\quad} x\odot y=x^{\ln(y)}.
\]
\begin{parts}
\part Is $\mathbb{R}^+$ a ring under these operations? 
\part Does $\mathbb{R}^+$ have an additive identity? If so, can you find the additive inverse of $7$?
\part Does $\mathbb{R}^+$ have an multiplicative identity? If so, can you find the multiplicative inverse of $7$?
\end{parts}


\question \textbf{M.20}
Solve each equation below. If no solution exists, explain why.
\begin{parts}
\part $[9]_{12}\cdot x=[0]_{12}$
\part $[11]_{12}\cdot x=[2]_{12}$
\part $[9]_{20}\cdot x=[0]_{20}$
\part $[99]_{100}\cdot x=[26]_{100}$
\end{parts}


\question \textbf{M.21}
Exactly one of the three rings below is a field. Identify which one is the field and then demonstrate a property of fields that is not satisfied for each of the other two rings.
\begin{itemize}
\item $\mathbb{Z}_{23}$
\item $\mathbb{Z}_{25}$
\item $\mathbb{Z}_{27}$
\end{itemize}


\question \textbf{M.22}
Let $\mathbb{Z}^*$ be a number system consisting of the set of all integers, with addition $(\oplus)$ and multiplication $(\otimes)$ defined as follows:
\[
x\oplus y= x+y+2\text{\quad and \quad}x\otimes y=xy+2x+2y+2.
\]
Show that $\mathbb{Z}^*$ is a commutative ring with identity under these operations.

\question \textbf{M.23}
Let $n$ be a natural number. Let $n\mathbb{Z}$ be the subset of $\mathbb{Z}$ consisting of all multiples of $n$. Determine whether  $n\mathbb{Z}$ a ring under the usual integer addition and multiplication. Is multiplication commutative? Does $n\mathbb{Z}$ have a multiplicative identity?







\end{questions}
\end{document}
