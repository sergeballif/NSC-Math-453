%\title{Math 453 HW 10}
\documentclass[addpoints]{exam}

\usepackage{amsmath,amssymb,enumitem,wrapfig,booktabs}
\usepackage{tikz}

\newcommand{\StudentName}{Student Name}
\newcommand{\AssignmentName}{HW 10}

\pagestyle{headandfoot}
\runningheadrule
\firstpageheadrule
\firstpageheader{Math 453}{\StudentName}{\AssignmentName}
\runningheader{Math 453}{\StudentName}{\AssignmentName}
\firstpagefooter{}{}{}
\runningfooter{}{}{}

\printanswers

\begin{document}


Organize your work and show any work that you want credit for. Use full sentences where possible.

\begin{questions}

\question \textbf{CORE M.31}
Determine whether the function $f\colon\mathbb{Z}_{3}\oplus\mathbb{Z}_4\to\mathbb{Z}_{12}$ given by $f([a],[b])=[4a+9b]$ is a ring isomorphism. (You may assume the function is well-defined).

\question \textbf{CORE M.32}
Determine whether the rings $\mathcal{P}_3$ and $\mathbb{Z}_2\times\mathbb{Z}_2\times\mathbb{Z}_2$ are isomorphic. If they are isomorphic, then find an isomorphism. If no isomorphism exists determine why using an invariant or other property.

\question \textbf{CORE M.33}
Determine with proof whether each  property of a ring $R$ below is an invariant.
\begin{parts}
\part $R$ only has exactly 2 units.
\part $R$ contains exactly 8 elements.
\part The characteristic of $R$ is $10$.
\part $R$ does not contain any negative numbers.
\end{parts}


\end{questions}
\end{document}
