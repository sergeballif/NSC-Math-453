%\title{Math 453 HW 6}
\documentclass[addpoints]{exam}

\usepackage{amsmath,amssymb,enumitem,wrapfig,booktabs}
\usepackage{tikz}

\newcommand{\StudentName}{Student Name}
\newcommand{\AssignmentName}{HW 6}

\pagestyle{headandfoot}
\runningheadrule
\firstpageheadrule
\firstpageheader{Math 453}{\StudentName}{\AssignmentName}
\runningheader{Math 453}{\StudentName}{\AssignmentName}
\firstpagefooter{}{}{}
\runningfooter{}{}{}

\printanswers

\begin{document}


Organize your work and show any work that you want credit for. Use full sentences where possible.

\begin{questions}

\question \textbf{CORE M.16}
For each of the number systems below determine whether each property in the table holds. Provide full justification for each answer. Then fill in the table with \textsf{N}'s and \textsf{Y}'s to summarize your answers.

\begin{parts}
\part $3\mathbb{Z}=\{\ldots,-6,-3,0,3,6,\ldots\}$ is the set of all multiples of $3$ with the usual addition and multiplication.
\begin{solution}

\end{solution}
\part $\mathbb{Q}^+\cup\{0\}$ is the set of all non-negative rational numbers with the usual addition and multiplication.
\part $R=\{[0]_{10},[2]_{10},[4]_{10},[6]_{10},[8]_{10}\}\subset\mathbb{Z}_{10}$.
\part $S=\mathbb{Z}$ with the operations of $\oplus$ (addition) and $\odot$ (multiplication) defined by $a\oplus b=a+b-1$ and $a\odot b=ab-1$.
\end{parts}

% Please add the following required packages to your document preamble:
% \usepackage{booktabs}

\begin{tabular}{lllll}
\toprule
Property                                                                       & $3\mathbb{Z}$ & $\mathbb{Q}^+\cup\{0\}$ & $R$ & $S$\\ \midrule
1. The system is closed under addition.   &  Y  &   &   &  \\ 
2. The system is closed under multiplication. &   &   &   &  \\  
3. Addition is associative.  &   &   &   &  \\ 
4. Multiplication is commutative.  &   &   &   &  \\ 
5. Multiplication distributes over addition.  &   &   &   &  \\  
6. There is an additive identity. &   &   &   &  \\ 
7. Each element has an additive inverse. &   &   &   &  \\  
8. There is a multiplicative identity. &   &   &   &  \\ 
9. Every nonzero element has a multiplicative inverse. &   &   &   &  \\ 
10. The zero product property of multiplication holds. &   &   &   &  \\ 
11. There are no zero divisors. &   &   &   &  \\ 
12. Additive Cancellation holds. &   &   &   &  \\  
13. Multiplicative cancellation (of nonzero elements) holds. &   &   &   &  \\ 
14. The ordering axioms hold.&   &   &   &  \\  \bottomrule
\end{tabular}


\question \textbf{M.17}
Answer the following questions about the power set $\mathcal{P}_5$.
\begin{parts}
\part What is the multiplicative identity element of $\mathcal{P}_5$?
\part What is the additive identity element of $\mathcal{P}_5$?
\part Find all of the units of $\mathcal{P}_5$.
\part Compute $\{1,2,3,4\}+\{2,3,4,5\}$ and $\{1,2,3,4\}\cdot\{2,3,4,5\}$
\part Find the additive inverse of $\{2,3,4\}$.
\end{parts}





\end{questions}
\end{document}
